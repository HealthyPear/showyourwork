% Define document class & import showyourwork
\documentclass[twocolumn]{aastex631}
\usepackage{showyourwork}


% Begin!
\begin{document}

% Author list
\author{{ "{" }}{{ author }}{{ "}" }}

% Abstract
\begin{abstract}
    I did something amazing.
\end{abstract}

% Main body
\section{Introduction}
Let me tell you all about it.

% A sample figure
\begin{figure}[ht!]
    \begin{centering}
        \includegraphics[width=0.4\linewidth]{mandelbrot.pdf}
        \includegraphics[width=0.4\linewidth]{koch.pdf}
        \caption{
            A sample plot with two subpanels.
        }
        \label{fig:fractals}
    \end{centering}
\end{figure}

\end{document}

% Custom stuff for this article
\usepackage{dirtree}
\usepackage{enumitem}
\usepackage{etoolbox}
\makeatletter
\patchcmd{\@verbatim}
  {\verbatim@font}
  {\verbatim@font\footnotesize}
  {}{}
\makeatother
\renewcommand{\descriptionlabel}[1]{{ "{%" }}
  \hspace\labelsep \upshape\bfseries #1%
}

% Begin!
\begin{document}

% Article title
\title{An open-source scientific article}

% Author list
\author{{ "{" }}{{ author }}{{ "}" }}

% Abstract
\begin{abstract}
    This is a sample article generated automatically by \texttt{showyourwork}, a workflow for easily producing open-source, reproducible scientific papers.
    The workflow takes as input a \LaTeX\, manuscript file and a \texttt{conda} environment specifying the requirements to build all of the figures in the paper from scratch.
    The output is the PDF you are reading right now, which is enhanced with clickable icons next to each of the figures that link to the exact version of the scripts that generated them.
    The color of these icons indicate the open-source ``status'' of the paper, ranging from \GitHubIconRed (issues with version control) to \GitHubIconBlue (tip top shape).
    The \texttt{GitHub} icon in the right margin of the abstract will take you to the  repository containing the source code for this paper.
    The icon next to it (if present) shows the status of the build on \texttt{GitHub Actions} and links to the log file for the workflow that generated this paper.
\end{abstract}

% Main body
\section{Introduction}
The \texttt{showyourwork} workflow is an attempt to tackle the ongoing reproducibility crisis in the natural sciences \citep[see, e.g.,][]{Baker2016}.
The idea behind \texttt{showyourwork} is simple: to make it easy for authors to publish the code that generated the results (and in particular the figures) in a scientific article.
This workflow ensures that the compiled article PDF is always in sync with the code that was used to generate its results.
It does this automatically---and seamlessly---with the help of the \texttt{Snakemake} workflow management system \citep{Molder2021}, the \href{https://github.com/tectonic-typesetting/tectonic}{tectonic} typesetting engine, and the \href{https://github.com/features/actions}{GitHub Actions} Continuous Integration/Deployment tool.

The \texttt{showyourwork} workflow acts on \texttt{GitHub} repositories with the following basic structure:
%
\vspace{1em}
%
\dirtree{{ "{%" }}
.1 repo/.
.2 figures/.
.3 *.py.
.2 tex/.
.3 ms.tex.
.3 bib.bib.
.2 environment.yml.
}
\vspace{1em}\noindent
%
The files \texttt{ms.tex} and \texttt{bib.bib} are the usual \LaTeX\, manuscript and bibliography files, \texttt{*.py} are \texttt{Python} scripts that generate the figures in the paper, and \texttt{environment.yml} is a \texttt{conda} environment file, specifying all of the dependencies needed to build the paper from scratch.

\begin{figure}[ht!]
    \begin{centering}
        \includegraphics[width=0.4\linewidth]{mandelbrot.pdf}
        \includegraphics[width=0.4\linewidth]{koch.pdf}
        \caption{
            A sample plot with two subpanels showing fractals.
            \emph{Left}: The Mandelbrot set.
            \emph{Right}: A Koch snowflake.
            Both figures were generated from the \texttt{Python} file \texttt{figures/fractal.py}, which is linked to in the GitHub icon in the right margin.
        }
        \label{fig:fractals}
    \end{centering}
\end{figure}

The article you are reading was built from a repository with a single figure script, \texttt{fractals.py}, which generated the two plots shown in Figure~\ref{fig:fractals}.
One of the main features of \texttt{showyourwork} is that it figures out all figure dependencies automatically, with intelligent caching that prevents re-running code that hasn't changed.
Here is the code snippet that created the figure:
%
\begin{verbatim}
\begin{figure}[t!]
    \begin{centering}
        \includegraphics[...]{mandelbrot.pdf}
        \includegraphics[...]{koch.pdf}
        \caption{...}
        \label{fig:fractals}
    \end{centering}
\end{figure}
\end{verbatim}
%
Note that we didn't have to add any extra code; \texttt{showyourwork} was able to inspect the figure label and the \verb+\includegraphics+ calls to infer that a script named \texttt{fractals.py} generates two figure files: \texttt{mandelbrot.pdf} and \texttt{koch.pdf}.
Whenever you label a figure with \verb+{fig:xxx}+, \texttt{showyourwork} assumes there's a file called \texttt{figures/xxx.py} that produces the required figures.

In addition to automatically running the figure scripts, \texttt{showyourwork} adds a \GitHubIcon to the margin next to the figure caption.
This is a clickable icon that links to the exact version of the script on \texttt{GitHub} used to generate the figure.
The color of the icon indicates the version control status:
%
\begin{center}
    \begin{tabular}{ r l }
        \GitHubIconRed    & The \texttt{Python} figure script does not exist                 \\
        \GitHubIconOrange & The \texttt{Python} figure script is not tracked by \texttt{git} \\
        \GitHubIconYellow & The \texttt{Python} figure script has uncommitted changes        \\
        \GitHubIconBlue   & The \texttt{Python} figure script is up to date
    \end{tabular}
\end{center}
%
The \GitHubIcon at the very top, next to the abstract, summarizes the overall version control status of all the figures in the paper.
If it's \GitHubIconBlue, everything is in order and your paper is fully reproducible.
Just always remember to commit your changes and push to \texttt{GitHub}.
This will automatically trigger the build on \texttt{GitHub Actions}, which builds your article from scratch every time (but with intelligent caching).
When that is done, the compiled PDF will have an extra icon in the margin next to the abstract indicating the successful online build, with a link to the complete log file (more details below).

\section{Using \texttt{showyourwork}}
%
There are a few different ways to use \texttt{showyourwork} in your research.

\subsection{Quick-and-dirty}
%
Create a new \texttt{GitHub} repository based on the template that generated this article by clicking \href{https://github.com/rodluger/showyourwork-example/generate}{here}.
Edit the manuscript and the figure script(s) as needed and push your changes to your repo.
\textbf{That's it!}
Your paper will automatically be built on \texttt{GitHub Actions} and will be available as the article file \texttt{ms.pdf} in the \texttt{main-pdf} branch of your repo and as a workflow artifact in the \texttt{Actions} tab on your repository \texttt{GitHub} page.
It will be updated every time you push new changes to your remote repo.

\subsection{Custom setup and local builds}
%
For a bit more fine-grained control, and to enable local builds of your article, install the \texttt{showyourwork} package, which requires the \href{https://www.anaconda.com/products/individual}{conda} package manager:
%
\begin{verbatim}
conda install -c conda-forge -c bioconda \
    mamba pip snakemake tectonic
python -m pip install \
    git+https://github.com/rodluger/showyourwork.git@main
\end{verbatim}
%
Please note that this is still the development version; soon you'll be able to install \texttt{showyourwork} directly from \texttt{conda}.
Once installed, you can run
%
\begin{verbatim}
showyourwork --new
\end{verbatim}
%
to create a fresh article repository in the current working directory.
You will be prompted for some information about the repo. Make changes to the files as needed, then run
%
\begin{verbatim}
showyourwork
\end{verbatim}
%
in the top level of the repository to build your paper.
This will generate the compiled PDF, \texttt{ms.pdf}, in the same directory.
If this is the first time you're running it, \texttt{showyourwork} will build all your figure scripts before compiling the PDF.
Once they are built, \texttt{showyourwork} caches the output to avoid re-running scripts that have not changed.

To enable remote builds, just initialize \texttt{git} in the repo, commit your changes, and push to a remote repository on \texttt{GitHub}:
%
\begin{verbatim}
git init
git add .
git commit -m "Initial commit"
git branch -M main
git remote add origin https://github.com/[user]/[repo].git
git push -u origin main
\end{verbatim}
%
As before, this will trigger a build on \texttt{GitHub Actions}; when that's done, your article will be aviable as the file \texttt{ms.pdf} in the \texttt{main-pdf} branch of your repo and as a workflow artifact in the \texttt{Actions} tab on your repository \texttt{GitHub} page.

\section{FAQs}
%
The \texttt{showyourwork} workflow is still a work in progress, and we're working on better documentation and on new features.
In the meantime, here are some frequently asked questions that may be helpful for first time users:

\begin{description}[style=nextline]
    \item[How do I include figure files directly?]
    While we strongly recommend making \emph{all} of your figures open source, sometimes this is not possible in practice, such as when your figure is a flowchart or an informational graphic. In this case, just commit the figure file directly (you might need to \texttt{--force} option when calling \texttt{git add}) and label it inside the \texttt{figure} environment with an asterisk:
    %
    \begin{verbatim}
    \begin{figure}[t!]
        \begin{centering}
            \includegraphics[...]{flowchart.pdf}
            \caption{...}
            \label{fig*:flowchart}
        \end{centering}
    \end{figure}
    \end{verbatim}
    %
    This will prevent \texttt{showyourwork} from trying to run a script to produce that figure.
    %
    \item[How do I include datasets?]
    Experimental or observational datasets can be included in the \texttt{data} directory. Note that you'll have to \texttt{--force} add them, since they are not tracked by default.
    If they are larger than about 1 MB, we recommend you instead host them externally on \href{https://zenodo.org/}{Zenodo} and download them in your figure scripts.
    We'll add a feature to cache external datasets on the \texttt{GitHub Actions} runner soon.
    %
    \item[What about support for other languages?]
    Currently only \texttt{Python} scripts are supported, but we're hoping to add support for other languages (\texttt{Julia}, \texttt{bash}, etc.) soon.
    We'll also soon support \texttt{Jupyter Notebooks}.
    %
    \item[How do I customize the workflow?]
    The \texttt{showyourwork} workflow uses \texttt{Snakemake} to build a simple dependency graph for the article: the article PDF depends on the figure files, which in turn depend on the figure scripts.
    Sometimes it's useful to specify additional dependencies, such as when a script imports a function from a different script, or when a script depends on a dataset that can potentially change.
    In that case, the user must tell \texttt{Snakemake} about this dependency so that it knows to re-run the figure script (and re-compile the paper) when, e.g., a dataset changes. This feature is under development and will be implemented soon!
    %
    \item[What if I don't use includegraphics?]
    That's fine! Common alternatives like \verb+\plotone+ call \verb+\includegraphics+ under the hood, so everything should still work.
    You can also alias the \verb+\includegraphics+ command however you'd like --- \texttt{showyourwork} will still be able to determine all the figure dependencies, as long as you label your figures correctly.
\end{description}

If you still have questions or feature requests, please \href{https://github.com/rodluger/showyourwork/issues}{open an issue}!

% Bibliography
\bibliography{bib}

\end{document}

