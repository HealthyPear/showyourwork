% !TeX root = ./ms.tex
\documentclass[modern]{aastex62}
\usepackage{showyourwork}

% Bibliography stuff
\bibliographystyle{aasjournal}

% Begin!
\begin{document}

% Title
\title{Show Your Work!}

% Author list
\author[\myORCID]{\myName}
\email{\myEmail}
\affil{\myAffil}

% Abstract
\begin{abstract}
    This is a sample article generated by the \showyourwork package, a workflow for producing open-source, fully reproducible scientific papers.
    The icons in the right margin point to the exact version of the repository that generated this PDF and to the directory containing tests that ensure the validity of the math in the paper.
    The color of the GitHub icon (\GitHubIcon) indicates the open-source ``status'' of the paper, ranging from \GitHubIconRed (issues with version control) to \GitHubIconBlue (tip top shape), and the pass/fail badge indicates whether the tests all passed (\TestPassIcon) or if one or more tests failed (\TestFailIcon).
\end{abstract}

% The rest of the paper
\section{Introduction}

The \showyourwork package is part of an attempt to tackle (at least in part) the ongoing reproducibility crisis in the natural sciences \citep[see, e.g.,][]{Baker2016}.
The idea behind this package is simple: to make it easy for authors to publish the code that generated the results (and in particular the figures) in a scientific article.
This package ensures that the compiled article PDF is always in sync with the code that was used to generate its results.
It does this automatically---and seamlessly---with the help of the \texttt{Snakemake} workflow management system \citep{Molder2021}, the \href{https://journals.aas.org/aastex-package-for-manuscript-preparation/}{AASTeX} markup package, the \href{https://github.com/tectonic-typesetting/tectonic}{tectonic} typesetting engine, and the \href{https://github.com/features/actions}{GitHub Actions} Continuous Integration/Deployment tool.

\section{Examples}

\begin{figure}[t!]
    \begin{centering}
        \includegraphics[width=0.5\linewidth]{heart.pdf}
        \caption{
            A sample plot showing the parametric ``heart curve'' given by Equation~(\ref{eq*:heart}).
            This figure was automatically generated from the \texttt{Python} file \texttt{figures/heart.py}, which is linked to in the GitHub icon in the right margin.
            The color of the margin icon indicates its status; see text for details.
        }
        \label{fig:heart}
    \end{centering}
\end{figure}

Here is the parametric equation for the heart in Figure~\ref{fig:heart}:
%
\begin{align}
    \label{eq*:heart}
    x & = 16 \sin^3 t                          \nonumber \\
    y & = 13 \cos t - 5 \cos\left(2 t\right) -
    2 \cos\left(3 t\right) - \cos\left(4 t\right)
\end{align}
%
for $t \in [-1, 1]$.

\begin{figure}[t!]
    \begin{centering}
        \includegraphics[width=0.45\linewidth]{fractal_mandelbrot.pdf}
        \includegraphics[width=0.45\linewidth]{fractal_koch.pdf}
        \caption{
            A sample plot with two subpanels showing fractals. \emph{Left}: The Mandelbrot set. \emph{Right}: A Koch snowflake. Both figures were generated from the \texttt{Python} file \texttt{figures/fractal.py}, which is linked to in the GitHub icon in the right margin.
        }
        \label{fig:fractal}
    \end{centering}
\end{figure}

Here is an equation associated with a ``unit test'':
%
\begin{align}
    \label{eq:euler}
    e^{i\pi} + 1 = 0
\end{align}
%
This equation has two margin icons: the usual GitHub icon (\GitHubIcon), which points to the associated test script \texttt{tests/test\_euler.py}, and a badge indicating whether the test passed
(\TestPassIcon) or failed (\TestFailIcon).

% Bibliography
\clearpage
\bibliography{bib}

\end{document}
